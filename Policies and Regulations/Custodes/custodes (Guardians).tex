\documentclass[10pt,a4paper]{report}
\usepackage[margin=2cm]{geometry} %to modify page layout
\usepackage{titlesec} %to hide chapter titles
\usepackage{enumitem} %to have control over how formatting on enumerated lists

\setlength{\parindent}{0pt}

%because it's current year and clickable table of contents is needed
\usepackage{hyperref}
\hypersetup{
	colorlinks,
	citecolor=black,
	filecolor=black,
	linkcolor=black,
	urlcolor=black
}

\usepackage[latin1]{inputenc}
\usepackage{amsmath}
\usepackage{amsfonts}
\usepackage{amssymb}
\usepackage{graphicx}


\title{Computer Science Students Club Policy}
\author{Custodes}
\date{19-02-2026}


%To hide these full page chapter title page things which are just a waste of paper
\titleformat{\chapter}[display] {\normalfont\bfseries}{}{0pt}{\Large}


\begin{document}
	\maketitle
	\newpage
	\begin{small}
		\tableofcontents
	\end{small}
	\newpage
	\chapter{Description}
		\section{Creation of Subcommittee}
            This document formally creates and outlines the functions and responsibilities of the Computer Science Students' Club's door subcommittee, formally known as Custodes. 
        
        \section{Purpose}
            Members of the subcommittee exist to open and hold the club room in the absence of an executive. Subcommittee members are allowed access to any club room keys. They are also responsible for the management and maintenance of the clubroom in the absence of an executive.   
    
		\section{Qualifications of Members}
			Members of the door subcommittee must:
			\begin{enumerate}
				\item Be considered trustworthy by the current committee.
				\item Be current student's of the University of Western Australia.
                \item Be fully privileged members of the Computer Science Students' Club. 
				\item Be willing to accept and carry out all instructions given to them by committee.
                \item Expectation to be active and present in the club reasonably frequently. 
			\end{enumerate}
            
        \section{Creation and Removal of Members}
			All executive members of the committee are to immediately be considered members of the door subcommittee. Any fully privileged member of the club may be appointed or removed by the committee during a formal meeting. There are to be no more than 10 members of the subcommittee at a given time, this includes executive members. 

        \section{Term Limits}
        Once elected, members of the door subcommittee serve until the start of the next university semester unless otherwise removed for any reason by the committee. 
            
	\chapter{Responsibilities}
		\section{Main Responsibilities}
		\begin{enumerate}
			\item Custodes are expected to protect the members and property of the club within reason. 
			\item Custodes are responsible for the oversight of all activities in the clubroom while it is under their control.
			\item Custodes are to assist any person looking to become a member of the club.
            \item Custodes are to follow all opening and closing procedures of the room. 
            \item Custodes are required to keep all keys to the room secure and private. 
            \item Custodes are expected to be able to handle and track payments within the club. 
		\end{enumerate}
		\section{Room Policy}
			\begin{enumerate}
				\item There may only be people in the clubroom if a member of the door subcommittee is present to hold the room. 
                \item All people must leave the room if there is no appropriate person is present to hold the room. 
                \item Upon opening the clubroom all procedures must be followed, mainly declaring the room open on relevant channels. 
                \item When closing the room, all closing procedures must be followed by members of the door subcommittee. This includes cleaning the room, securing the room and declaring the room closed on relevant channels. 
                \item Individuals banned from the club or are of high-risk to members are not allowed to be present in the room. 
                \item Club members not on the door subcommittee may be able to temporarily hold the room if permitted by an executive of the club. The room may not be held for longer than two hours by a non door subcommittee member. This temporary custode is not to be given access to a key. 
			\end{enumerate}
			
	\chapter{Rights, Powers and Privileges}
		\section{Powers given by Committee to fulfill Subcommittee Purpose}
		
		\begin{enumerate}
            \item Members of the door subcommittee are allowed access to and are able to hold keys to the club door. Non executive members are not to hold keys for an extent of time which limits members access to the clubroom for an unreasonable amount of time. 
			\item Individuals on the door subcommittee are allowed the ability to process cash, online and eftpos payments within the clubroom. 
		\end{enumerate}
		
		\section{Additional Powers granted in specific circumstances}
		\begin{enumerate}
			\item In the circumstance where there is someone looking to speak to someone from the club, a Custode may take action to address this person as seen appropriate. Any action taken by the Custode must be noted down and sent to the committee
		\end{enumerate}
		
		\subsection{Temporary Removal of Individuals from the Room}
		In most cases, offenders should be given a warning before being removed. Members of the door subcommittee may remove someone from the room for any reason at their discretion. In the event that an offender refuses to abide by the removal, UWA Security can be called. This is not a permanent expulsion. 

    \section{Lockbox}
    \subsection{Usage}
    When a lockbox is present, any additional clubroom keys will be stored there. All members of the door subcommittee will be given access and are expected to keep the code private. It is expected for subcommittee members to store keys in the lockbox as opposed to on their person at all times. 
    \subsection{Combination}
    Only executive members or CSSE building administration are allowed to change the lockbox combination. It is expected the combination be changed after the removal of a member of the door subcommittee or start of a new university semester. 
    \subsection{Absence}
    In the absence of the a lockbox, any additional keys not held by the president of the club are to be held at CSSE reception and only approved members are to sign them out. A list of approved members are to be provided to CSSE administration including name, student number and member position. 
		
		
	\chapter{Additional Important Information}
        \section{Items that Custodes should be aware of}
		\subsection{UWA Security Contact Details}
		\begin{itemize}
			\item Emergency: 6488 2222
			\item Non-Emergency: 6488 3020
		\end{itemize}
        \subsection{Payment Information}
        Know all relevant information to make payments.
        \subsection{Membership Signup and Validation}
		Know how to signup new members to the club and verify existing ones.
        
		\section{Expulsion}
		It may be seen necessary to permanently expel someone from the clubroom for a limited or extended amount of time. A formal expulsion requires a majority vote by members of the door subcommittee. 
		\begin{enumerate}
			\item A person who is expelled is forbidden entry to the clubroom for the duration of the expulsion.
			\item An expelled person may enter the clubroom with an acceptable reason under the oversight and with the permission of an executive member.
			\item An expulsion expires either once the conditions set by the committee have been met, or by vote at an Ordinary Committee Meeting.
			\item Formal documentation of the event is required, including a detailed reason for expulsion. 
		\end{enumerate}

\end{document}
